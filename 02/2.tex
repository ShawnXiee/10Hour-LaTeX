\documentclass[12pt]{article}


\begin{document}
\section{Inline \& display formula}
Suppose we are given a rectangle with side length $x$ and $x+1$, then the equation $A=x^2+x$ represents the area of the rectangle.

Suppose we are given a rectangle with side length \(x\) and \(x+1\), then the equation \(A=x^2+x\) represents the area of the rectangle.

Suppose we are given a rectangle with side length $x$ and $x+1$, then the equation $$A=x^2+x$$ represents the area of the rectangle.

Suppose we are given a rectangle with side length $x$ and $x+1$, then the equation 
\[
A=x^2+x
\] 
represents the area of the rectangle.

Suppose we are given a rectangle with side length $x$ and $x+1$, then the equation (\ref{eq:area}) represents the area of the rectangle.
\begin{equation}\label{eq:area}
A=x^2+x
\end{equation}


\section{Superscripts}
\[2x^3\]
\[2x^{34}\]
\[2x^{3x+4}\]
\[2x^{3x^4+5}\]

\section{Subscripts}
\[x_1\]
\[x_{12}\]
\[ x_{1_{2_3}}   \]

\section{Greek letters}
$\alpha$, $\beta$, $\gamma$, $\theta$

\section{Square root}
\[\sqrt{2}\]
\[\sqrt[3]{2}\]
\[\sqrt{x^2+y^2}\]
\[\sqrt{1+\sqrt{x}}\]


\section{Fractions}
Inline fractions $\frac{2}{3}$ and $\displaystyle{\frac{2}{3}}$ .

\[\frac{x}{x^2+x+1}\]

\[\frac{\sqrt{x+1}}{\sqrt{x-1}}\]

\[\frac{1}{1+\frac{1}{x}}\]

\[\sqrt{\frac{x}{x^2+x+1}}\]

\section{Brackets}
$$(x+1)$$
$$3[2+(x+1)]$$
$$\{a,b,c\}$$

$$3\left(\frac{2}{5}\right)$$
$$3\left[\frac{2}{5}\right]$$
$$3\left\{\frac{2}{5}\right\}$$

$$\left|\frac{x}{x+1}\right|$$

$$\left\{x^2\right.$$

$$\left. \frac{dy}{dx} \right|_{x=1}$$

\[
\Bigg(\bigg(\Big(\big( (
\]
\end{document}